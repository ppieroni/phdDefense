\section[Exposicion]{C\'alculo de la exposici\'on}

\begin{frame}
 \frametitle{Estrategia de cada b\'usqueda}
 \begin{center}
  \pgfimage[width=0.9\textwidth]{./fig/estrategiaAuger/analysisSchema_4}
 \end{center}
\end{frame}

\begin{frame}
 \frametitle{C\'alculo de la exposici\'on ${\cal E}(E_\nu)$}\footnotesize
 
 \begin{exampleblock}{Definici\'on}
%  \scriptsize
 \begin{displaymath}
 N_{esp}=\int\limits_{\mathbf{E_{\nu}}}~\Phi(E_{\nu}){\cal E}(E_\nu)~dE_\nu
 \end{displaymath}
 \end{exampleblock}

 \begin{block}{Exposici\'on a neutrinos DG}
 \scriptsize
	\begin{displaymath}
	 {\cal E}_{DG}(E_\nu)\equiv\iint\limits_{\mathbf{D}~\mathbf{\Omega}}
	 {\color<2->{Red}
	 P(D|E_{\nu},\Omega)~
	 }
	 {\color<3>{Green}
	 \left[~
	 \iint\limits_{T~A}\epsilon(E_{\nu},\Omega,D,\vec{r},t)~d\vec{r}~dt
	 \right]
	 }
	 ~d\Omega~dD
	\end{displaymath}
 \end{block}
 \begin{block}{Exposici\'on a neutrinos ES}
 \scriptsize
%   \footnotesize
	\begin{displaymath}
	 {\cal E}_{ES}(E_\nu)\equiv\iiint\limits_{\mathbf{E_\tau}~\mathbf{x_d}~\mathbf{\Omega}}
	 {\color<2->{Red}
	 P({\rm x_d},E_\tau|E_{\nu},\Omega)~
	 }
	 {\color<3->{Green}
	 \left[~
	 \iint\limits_{T~A}\epsilon(E_\tau,\Omega,{\rm x_d},\vec{r},t)~d\vec{r}~dt
	 \right]
	 }
	 ~d\Omega~d{\rm x_d}~dE_\tau
	\end{displaymath}
 \end{block}
 
 \begin{block}{Interpretaci\'on}
	\begin{itemize}
	\item<2-> {\color{Red} Probabilidad de que un neutrino inicie una lluvia.}
	\item<3> {\color{Green} Probabilidad de detectar la lluvia, integrada en tiempo y \'area.}
	\end{itemize}
 \end{block}
\end{frame}
% 
\begin{frame}{T\'erminos de probabilidad}
\footnotesize
 \begin{block}{Lluvias DG}
  \centering
	\begin{itemize}
	 \item Probabilidad de que un $\nu_x$ genere una lluvia a una profundidad $D$ en la atm\'osfera:
			$$
			P(D|E_{\nu},\theta) = \frac{1}{m}~\sigma(E_{\nu})\cos\theta
			$$
	\end{itemize}
 \end{block}
 
 \begin{textblock}{3.}(1.5,9)
  \scriptsize
  \begin{alertblock}{$f(E_\tau|E_\nu,\theta)$:}
  \centering
	Obtenida mediante Monte Carlo
  \end{alertblock}
 \end{textblock}
 
 \begin{block}{Lluvias ES}
  \begin{enumerate}
   \item Probabilidad de que un $\nu_\tau$ genere un $\tau$:
	\begin{center}
		\pgfimage[width=0.45\textwidth]{./fig/exposicionAuger/pdfs_defensa}
	\end{center}
   \item Probabilidad de que el $\tau$ decaiga en en la atm\'osfera y genere una lluvia:
   \begin{displaymath}
    h(x_d,(E_\tau,\theta))=
		\exp{\left(
		-\frac{x_d}{|\cos\theta|\lambda(E_\tau)}
		\right)}
		\frac{1}{|\cos\theta|\lambda(E_\tau)}
   \end{displaymath}
  \end{enumerate}
 \end{block}
\end{frame}



\begin{frame}{Exposici\'on combinada}
	\begin{alertblock}{}
		\begin{center}
			\textbf{La muestra de datos es escrutada con los tres criterios}
		\end{center}
	\end{alertblock}
	\begin{block}{M\'etodo de combinaci\'on}
		\begin{center}
		\pgfimage[width=0.75\textwidth]{fig/exposicionAuger/sketch_combined_5}
		\end{center}
	\end{block}
\end{frame}

\begin{frame}{Exposici\'on combinada}
	\begin{block}{Resultado}
		\begin{center}
		\pgfimage[width=0.75\textwidth]{fig/exposicionAuger/exposure_combined_ageing}
		\end{center}
	\end{block}
	\begin{alertblock}{}
		\begin{center}
			\textbf{El canal ES domina por debajo de $\mathbf{10^{19}{\rm\ \mathbf{eV}}}$}
		\end{center}
	\end{alertblock}
\end{frame}
% 

\begin{frame}{Exposici\'on combinada}
\framesubtitle{Escenario $\Phi\propto E^{-2}$}
	\begin{exampleblock}{Distribuci\'on de la exposici\'on}
		\begin{center}
		\begin{tabular}{|c|c|c|c|c|}
			\hline
			\diagbox{Lluvia}{Criterio} & ES & DGH & DGL  & Total\\ \hline
			ES     &    \alert<2>{0.80}& \alert<2>{0.04} & $<0.001$ & \alert<1>{0.84} \\ \hline
			DGH    &    \alert<3>{0.03}       &    \alert<3>{0.11}       &     $<0.001$ & \alert<1>{0.14} \\ \hline
			DGL    &    $<0.001$   &    $<0.001$   &     0.02     & \alert<1>{0.02} \\
			\hline
		\end{tabular}
		\end{center}
	\end{exampleblock}
	
	\begin{block}{}
	 \begin{itemize}[<+->]
	  \item Earth Skimming domina la detecci\'on
	  \item El criterio de DGH recupera un $4\%$ de exposici\'on a eventos ES
	  \item El criterio ES recupera un $3\%$ de exposici\'on a eventos DGH
	 \end{itemize}
	\end{block}

	
\end{frame}

\begin{frame}{Errores sistem\'aticos}
	\begin{block}{Errores sistem\'aticos}
		\begin{center}
			\renewcommand{\arraystretch}{1.4}
			\scriptsize
% 			\newcolumntype{C}[1]{>{\centering\arraybackslash}p{#1}}
				\begin{tabular}{|C{0.2\textwidth}|c|c|c|c|}
% 				\begin{tabular}{|l|c|c|c|c|}
				\hline
				Fuente del  & ES        & DGH       & DGL        & Combinaci\'on         \\
				sistemático & ($90^\circ,95^\circ$) & ($75^\circ,90^\circ$) & ($65^\circ,75^\circ$) & ES / DGH / DGL   \\
		% 		\hline
		% 		& {\tiny \bf GAP 2013-100}     & \multirow{2}{*}{\tiny \bf PRD 84, 2011}    &   \multirow{2}{*}{\tiny \bf GAP2013-013} & \multirow{2}{*}{\tiny \bf 83.9\% / 13.7\% / 2.4\% }\\
		% 		& {\tiny \bf PRD 79, 2009}     &     &  &  \\
				\hline
				
				Gen. de interacción primaria    &  no apl. &   0\%, -7\%     &   +3\%, -4\%  & +0.07\%, -1.0\% \\
				
				\hline
				
				PDF en el generador             &  no apl. &   0\%, -7\%     &   +4\%, -5\%  & +0.1\%, -1.0\% \\
				
				\hline
				
				Simulador de EAS                &  no eval. &   0\%, -17\%    &   +17\%, 0\%  & +0.4\%, -2.3\% \\
				
				\hline
				
				Modelo hadrónico                & +4.7\%, -1\%      &  +5\%, -2\%     &   +0\%, -6\%  & +4.6\%, -1.3\% \\
				
				\hline
				Algoritmo de thinning           & +0.3\%, 0\%   &  +7\%,  0\%     &   +7\%,  0\%  & +1.1\%, -0.0\% \\
				\hline
				\hline
				\multirow{2}{*}{\bf $\bm{ \sigma_{\nu_\tau}\ \otimes\ \tau}$ E-loss}    & \multirow{2}{*}{\textcolor{Red}{+40\%, -33\%}}  & \multirow{2}{*}{+9\%, -9\%}  & \multirow{2}{*}{+7\%, -7\%} & \multirow{2}{*}{\bf +35\%, -29\%} \\
													&                 &                 &             & \\
				\hline
				\hline
		% 				%%%%%%%%%%%%%%%%%%%%%%%%%%%%%%%%%%%%%%%%%%%%%%%%%%%%%%%%%%%%%%%%%%%%%%%%%%%%%%%%%%%%%%%%%%%%%%%%%%
				Topography          &  +18\%, 0\%    & +24\%, 0\% & not apl.   & +18\%, 0\%  \\
				\hline
				\hline
				{\bf Total}                     &  \multicolumn{3}{c|} {}  & {\bf +39\%, -29\%}         \\
				\hline
				%%%%%%%%%%%%%%%%%%%%%%%%%%%%%%%%%%%%%%%%%%%%%%%%%%%%%%%%%%%%%%%%%%%%%%%%%%%%%%%%%%%%%%%%%%%%%%%%%%
				\end{tabular}
		\end{center}
	\end{block}
% 	\begin{exampleblock}{}
% 		\begin{center}\footnotesize
% 			La combinaci\'on se obtuvo teniendo en cuenta la contribuci\'on relativa de cada canal a la exposici\'on y correlaci\'on total.
% 		\end{center}
% 	\end{exampleblock}
\end{frame}