\section[Conclusiones]{Conclusiones}

\begin{frame}{Comentarios finales}
\scriptsize
	\begin{block}{}
	\centering
	\textbf{Se estudi\'o la medici\'on de neutrinos c\'osmicos ultra energ\'eticos con detectores de superficie.}
	\end{block}
	
	\begin{exampleblock}{\centering B\'usqueda de neutrinos con el observatorio Pierre Auger}
	\begin{itemize}[<alert@+|+->]
	 \item Se presentaron los procedimientos de identificaci\'on de lluvias inclinadas y j\'ovenes, y el c\'alculo de la exposici\'on asociada a las tres b\'usquedas realizadas por Auger.
	 \item Se integraron los canales de b\'usqueda en un an\'alisis combinado, teniendo en cuenta la correlaci\'on entre sistem\'aticos y las posibilidades de detecci\'on cruzada.
	 \item Se obtuvo el l\'imite integrado m\'as estricto al flujo difuso en el rango \cant{1\text{-}80}{EeV}, \cant{k<6.4\times 10^{-9}}{Gev\ cm^{-2}\ s^{-1}\ sr^{-1}} (90$\%$ C.L.) con $\Phi=k\ E^{-2}$
	 \item Se rechazaron los modelos de flujos que suponen protones como primario y una evoluci\'on optimista para las fuentes.
	\end{itemize}
	\end{exampleblock}
	
	\begin{alertblock}{\centering B\'usqueda de neutrinos con un arreglo de antenas de radio}
	\begin{itemize}[<alert@+|+->]
	 \item Se describi\'o la emisi\'on de ondas de radio por lluvias atmosf\'ericas y se realiz\'o una caracterizaci\'on de la huella dejada en la superficie por neutrinos rasantes.
	 \item Se determin\'o la eficiencia y la exposici\'on para distintas configuraciones de antenas.
	 \item Se encontr\'o que para un detector del tama\~no de GRAND es factible alcanzar una exposici\'on por unidad de tiempo un orden de magnitud mayor que la de Auger.
	 \item Esto permitir\'ia estudiar la mayor\'ia de los modelos que utilizan protones como primario e incluso alcanzar los generados por hierro.
	\end{itemize}
	\end{alertblock}

\end{frame}

\begin{frame}{}
 \begin{alertblock}{}
  \centering
  \huge{
  Muchas gracias!
  }
 \end{alertblock}
\end{frame}

\begin{frame}{Agradecimientos}
 \begin{alertblock}{}
  \begin{itemize}
   \item Fernanda
   \item Ricardo y Jaime
   \item Yann y Javier
   \item Mi familia
   \item Mis amigos
   \item Compa\~neros de trabajo
   \item La UBA
  \end{itemize}

 \end{alertblock}
\end{frame}