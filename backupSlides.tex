\section{Backup Slides}

\frame{\sectionpage}

	
\begin{frame}
\frametitle{Observatorio Pierre Auger (PAO)}
\framesubtitle{Descripci\'on}
	\begin{block}{Detector - Tama\~no}
		\begin{center}
			\pgfimage[width=0.7\textwidth]{./fig/detectorAuger/augercapfed}
		\end{center}
	\end{block}
\end{frame}

\begin{frame}{Identificaci\'on de neutrinos con el SD de Auger}
	\begin{alertblock}{}\centering
	Con el SD es posible distinguir frentes mu\'onicos de frentes electromagn\'eticos.
% 			With the SD, we can distinguish muonic from electromagnetic shower fronts (using the time structure of the signals in the water Cherenkov stations).
	\end{alertblock}
	
	\begin{block}{Traza Cherenkov medida con una resoluci\'on de 25ns}
	\centering
	\pgfimage[width=0.8\textwidth]{fig/detectorAuger/tanque_muon.pdf}
	\end{block}
	
\end{frame}

\begin{frame}
 \frametitle{Divisi\'on de las muestras para DG}
	\begin{block}{DGL}
		\begin{center}
		\pgfimage[width=0.9\textwidth]{./fig/simulacionAuger/periodosDGL}
		\end{center}
	\end{block}
	\begin{block}{DGH}
		\begin{center}
		\pgfimage[width=0.9\textwidth]{./fig/simulacionAuger/periodosDGH}
		\end{center}
	\end{block}
\end{frame}

\begin{frame}
 \frametitle{Divisi\'on de las muestras para ES}
	\begin{block}{ES}
		\begin{center}	 
		\pgfimage[width=0.9\textwidth]{./fig/simulacionAuger/periodosES}
		\end{center}
	\end{block}
\end{frame}


\begin{frame}
 \frametitle{Estrategia de cada b\'usqueda}
 \begin{center}
  \pgfimage[width=0.9\textwidth]{./fig/estrategiaAuger/analysisSchema_1}
 \end{center}
\end{frame}


\begin{frame}
 \frametitle{Selecci\'on de lluvias j\'ovenes}
 \begin{block}{Discriminante de Fisher}
 \centering
  \pgfimage[width=0.85\textwidth]{fig/seleccionAuger/ideaFisher}
 \end{block}
\end{frame}

\begin{frame}{Eficiencias DG}
	\begin{block}{Disparo com funci\'on de la prifundidad}
		\begin{center}
		\pgfimage[width=0.75\textwidth]{fig/exposicionAuger/eff_10EeV_85}
		\end{center}
	\end{block}
\end{frame}

\begin{frame}{Eficiencias DG}
	\begin{block}{Disparo com funci\'on de la energ\'ia}
		\begin{center}
		\pgfimage[width=0.75\textwidth]{fig/exposicionAuger/eff_varios_85}
		\end{center}
	\end{block}
\end{frame}

\begin{frame}{Eficiencias DG}
\footnotesize
	\begin{block}{Disparo com funci\'on del \'angulo cenital}
		\begin{center}
		\pgfimage[width=0.4\textwidth]{fig/exposicionAuger/eff_1EeV_80}\hspace{3mm}
		\pgfimage[width=0.4\textwidth]{fig/exposicionAuger/eff_1EeV_85}
		\end{center}
	\end{block}
% \end{frame}
% 
% \begin{frame}{Eficiencias DG}
	\begin{block}{Disparo com funci\'on del canal de interaccion}
		\begin{center}
		\pgfimage[width=0.4\textwidth]{fig/exposicionAuger/eff_CCvsNC_85}\hspace{3mm}
		\pgfimage[width=0.4\textwidth]{fig/exposicionAuger/eff_tau_1EeV_85}
		\end{center}
	\end{block}
\end{frame}

\begin{frame}{Eficiencias ES}
	\begin{block}{Disparo com funci\'on de la altura de decaimiento}
		\begin{center}
		\pgfimage[width=0.75\textwidth]{fig/exposicionAuger/eff_18_8931_forThesis}
		\end{center}
	\end{block}
\end{frame}

\begin{frame}{Eficiencias ES}
	\begin{block}{Disparo com funci\'on de la energ\'ia}
		\begin{center}
		\pgfimage[width=0.75\textwidth]{fig/exposicionAuger/eff_multEnergy_forThesis}
		\end{center}
	\end{block}
\end{frame}

\begin{frame}{Eficiencias ES}
\footnotesize
	\begin{block}{Disparo com funci\'on del \'angulo cenital}
		\begin{center}
		\pgfimage[width=0.48\textwidth]{fig/exposicionAuger/eff_multiTheta_forThesis}
		\hspace{1mm}
		\pgfimage[width=0.48\textwidth]{fig/exposicionAuger/eff_multTheta_h10_forThesis}
		\end{center}
	\end{block}
	\begin{block}{Definicion de la variable $h_{10}$}
		\begin{center}
		\pgfimage[width=0.45\textwidth]{fig/exposicionAuger/hc_def.pdf}
		\end{center}
	\end{block}
\end{frame}

\begin{frame}{Integraci\'on temporal y en superficie}
	\begin{block}{M\'etodo de calculo}
		\begin{center}
		\pgfimage[width=0.6\textwidth]{fig/exposicionAuger/aperturaReal}
		\end{center}
	\end{block}
\end{frame}

\begin{frame}{Integraci\'on temporal y en superficie}
	\begin{block}{Selecci\'on de la configuraci\'on del detector}
		\begin{center}
		\pgfimage[width=0.75\textwidth]{fig/exposicionAuger/t2FilePlot}
		\end{center}
	\end{block}
\end{frame}


\begin{frame}{Integraci\'on temporal y en superficie}
	\begin{block}{Envejecimiento de detector}
		\begin{center}
		\pgfimage[width=0.9\textwidth]{fig/exposicionAuger/timeEvolution_1}<1>
		\pgfimage[width=0.9\textwidth]{fig/exposicionAuger/timeEvolution_2}<2>
		\pgfimage[width=0.9\textwidth]{fig/exposicionAuger/timeEvolution_3}<3>
		\pgfimage[width=0.9\textwidth]{fig/exposicionAuger/fractionEvolution}<4>
		\end{center}
	\end{block}
\end{frame}


\begin{frame}{Integraci\'on temporal y en superficie}
	\begin{block}{Envejecimiento de detector}
		\begin{center}
		\pgfimage[width=0.55\textwidth]{fig/exposicionAuger/timedecay_vs_reflect_absorp_2}
		\end{center}
	\end{block}
	\begin{block}{Envejecimiento de detector}<2>
		\begin{center}
		\renewcommand{\arraystretch}{1.4}
		\footnotesize
		\begin{tabular}{|l|ccc|c|}
					\hline
					Período       & tyRef & wAbs & LDT        &    Pérdida de exposición \\
					\hline
					$2004 - 2008$ & 0.94  & 100  & $\sim63$ns &    $--$ \\
					$2009 - 2010$ & 0.94  & 80   & $\sim60$ns &    $-15.2\%$\\
					$2011 - 2013$ & 0.93  & 100  & $\sim57$ns &    $-17.5\%$\\
					\hline
		\end{tabular}
		\end{center}
	\end{block}
\end{frame}

\begin{frame}{Exposici\'on combinada}
	\begin{block}{Obtenci\'on de la exposicion total}
		\begin{center}
			\begin{displaymath}\renewcommand{\arraystretch}{2}
			\begin{array}{rcl}
			N_{esp}& =& N_{esp}^{DGL}+N_{esp}^{DGH}+N_{esp}^{ES} \\ 
			& = & \int\limits_{E_\nu}\Phi(E_\nu)~({\cal E}^{DGL}+{\cal E}^{DGH}+{\cal E}^{ES})(E_\nu)~dE_\nu\\
			& \equiv & \int\limits_{E_\nu}\Phi(E_\nu)~{\cal E}(E_\nu)~dE_\nu
			\end{array}
			\end{displaymath}
		\end{center}
	\end{block}
	
	\begin{alertblock}{}
	\centering
% 		\begin{center}
			\begin{displaymath}
			{\cal E}(E_\nu) = {\cal E}^{DGL}+{\cal E}^{DGH}+{\cal E}^{ES}
			\end{displaymath}
% 		\end{center}
	\end{alertblock}
	
	\begin{exampleblock}{}
		\begin{center}
			\textbf{La exposici\'on combinada es la suma de las individuales}
		\end{center}
	\end{exampleblock}
\end{frame}

\begin{frame}{Errores sistem\'aticos}
	\begin{alertblock}{Bines representativos en ES}
		\begin{center}
			\pgfimage[width=0.75\textwidth]{fig/exposicionAuger/importantBins_oldWeights}
		\end{center}
	\end{alertblock}
	\begin{block}{Resultado}
		\begin{center}
			Se recalculo la exposici\'on sobre los bines que m\'as contribuyen a la exposici\'on.
		\end{center}
	\end{block}
\end{frame}

\begin{frame}{Errores sistem\'aticos}
	\begin{block}{}
		\begin{center}
			\pgfimage[width=0.45\textwidth]{fig/exposicionAuger/nu_xsection_models}\hspace{2mm}
			\pgfimage[width=0.45\textwidth]{fig/exposicionAuger/tau_E_loss_models}
		\end{center}
	\end{block}
	\begin{exampleblock}{}
		\begin{center}
			\begin{tabular}{|l|l|l|}
			\hline
			\textbf{Modelo}      & Secci\'on eficaz& P\'erdida de energ\'ia del $\tau$ \\ 
			\hline
			Referencia &    Sarkar     & ALLM\\ 
			%
			Pesimista &  ASW &     BB\\ 
			%
			Optimista &   Armesto Sat. $\lambda=0.4$&  ASW\\
			\hline 
			\end{tabular}  
		\end{center}
	\end{exampleblock}
\end{frame}

\begin{frame}{Errores sistem\'aticos}
	\begin{block}{Cambio en las probabilidades}
		\begin{center}
			\pgfimage[width=0.75\textwidth]{fig/exposicionAuger/pdfSyst}
		\end{center}
	\end{block}
	\begin{exampleblock}{Resultado}
		\begin{center}
			Cambian la probabilidad de emerger y el espectro de energ\'ia.
		\end{center}
	\end{exampleblock}
\end{frame}


\begin{frame}{Emisi\'on de radio en lluvias atmosf\'ericas}
\framesubtitle{Origen}
\footnotesize
	
	\begin{block}{Part\'icula que se desplaza en l\'inea recta a velocidad constante}
		\begin{center}
		\pgfimage[width=0.5\textwidth]{fig/EASRadio/trackSch}
		\end{center}
	\end{block}
	
	\begin{block}{Aproximaci\'on ZHS}
		\begin{center}
		\pgfimage[width=0.5\textwidth]{fig/EASRadio/zhs_pulse}
		\end{center}
	\end{block}
\end{frame}

\begin{frame}{Ubicaci\'on del detector}
\footnotesize
	\begin{block}{Posibles ubicaciones}
		\begin{center}\scriptsize
		\begin{tabular}{LLLL}
		\toprule
		Experimento (Sitio) & Declinaci\'on (+E,-W) & Inclinaci\'on (+D,-U)& Intensidad [Gauss] \\
		\midrule
		Auger AERA (Malarg\"ue Argentina) 
% 		& $35^\circ12$'${\rm S} $ $69^\circ18$'${\rm W}$
		& $(1.68\pm0.37)^\circ$ & $(-36.3\pm0.22)^\circ$ & $0.2392\pm0.0015$ \\ \midrule
		\alert<2>{Tunka Rex  (Tunka Valley Rusia)}
% 		& $51^\circ48$'${\rm N}$ $103^\circ04$'${\rm E}$
		& \alert<2>{$(-2.82\pm0.37)^\circ$} & \alert<2>{$(-71.2\pm0.22)^\circ$} & \alert<2>{$0.6037\pm0.0015$} \\ \midrule
		Trend  (Tian shan China) 
% 		& $40^\circ32$'${\rm N}$ $78^\circ25$'${\rm E}$
		& $(3.76\pm0.37)^\circ$ & $(60.0\pm0.22)^\circ$ & $0.5391\pm0.0015$ \\
		\bottomrule
		\end{tabular}
		\end{center}
	\end{block}
	\begin{exampleblock}{Caracter\'isticas deseables}
	\begin{itemize}
	 \item Verticalidad
	 \item Intensidad 
	\end{itemize}
	\end{exampleblock}

\end{frame}

\begin{frame}{C\'alculo de la eficiencia}
\footnotesize
	\begin{block}{Se lanz\'o cada lluvia 1000 veces sobre la celda primitiva}
	\centering
		\pgfimage[width=0.48\textwidth]{fig/resultadosRadio/17.00_89.90_00.00_00025_01238_1000_1000_90_re}\hspace*{2mm}
		\pgfimage[width=0.48\textwidth]{fig/resultadosRadio/17.00_89.90_00.00_00025_01238_1500_1500_60_re} \\ \vspace*{2mm}
		\pgfimage[width=0.48\textwidth]{fig/resultadosRadio/17.00_89.90_00.00_00025_01238_500_4000_90_de}\hspace*{2mm}
		\pgfimage[width=0.48\textwidth]{fig/resultadosRadio/17.00_89.90_00.00_00025_01238_750_750_60_hc}
	\end{block}
% 	\begin{block}{}
% 		\begin{center}
% 		
% 		\end{center}
% 	\end{block}
\end{frame}

\begin{frame}{Detector ideal}
\footnotesize
		\begin{alertblock}{Caracter\'isticas:}
		 \centering
		 \textbf{100$\bm\%$ eficiente en el rango angular $\bm{90^\circ}$ - $\bm{92.5^\circ}$}
		\end{alertblock}

		\begin{block}{\scriptsize L\'imite diferencial: 90000 antennas - Trigger local 50${\rm \mu Vm}$ - L = 250 ${\rm km}$ - 500${\rm km}$}
			\begin{center}
			\pgfimage[width=0.9\textwidth]{fig/resultadosRadio/limits_future_v1_1_2.pdf}
			\end{center}
		\end{block}
\end{frame}