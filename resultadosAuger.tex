\section[Resultados]{B\'usqueda de candidatos y resultados}

\begin{frame}
 \frametitle{Estrategia de cada b\'usqueda}
 \begin{center}
  \pgfimage[width=0.9\textwidth]{./fig/estrategiaAuger/analysisSchema_5}
 \end{center}
\end{frame}

\begin{frame}{Unblinding ES}
	\begin{center}
	\pgfimage[width=0.75\textwidth]{fig/resultadosAuger/Unblinding_ES_200613_mod}
	\end{center}
	\begin{block}{}
	\centering {\Large \color{red}\bf 0 candidatos}
	\end{block}
\end{frame}


\begin{frame}{Unblinding DGH}
	\begin{columns}[T]
	\column{.5\textwidth}
	\begin{tabular}{c}
	\pgfimage[width=\textwidth]{fig/resultadosAuger/DGH_Retrining_May2012_2_low_Nor_mod}\\
	\pgfimage[width=\textwidth]{fig/resultadosAuger/DGH_Retrining_May2012_2_med_Nor_mod}
	\end{tabular}

	\column{.5\textwidth}
	\pgfimage[width=\textwidth]{fig/resultadosAuger/DGH_Retrining_May2012_2_high_Nor_mod} 
	\vspace*{1cm}
	\begin{block}{}
	\centering \Large \bf \alert{0 candidatos}
	\end{block}
	\end{columns}
\end{frame}

\begin{frame}
 \frametitle{Estrategia de cada b\'usqueda}
 \begin{center}
  \pgfimage[width=0.9\textwidth]{./fig/estrategiaAuger/analysisSchema_6}
 \end{center}
\end{frame}

\begin{frame}{L\'imite al flujo de neutrinos hasta el 20 Jun 13}
	\scriptsize
		\begin{textblock}{15}(0.5,2.3)
			\begin{alertblock}{C\'alculo del l\'imite}
				\centering
				Asumiendo $\nu$ flux: $\Phi_\nu ~ = ~ k ~ E_\nu^{-2}$ $\Rightarrow$ $k_{\rm up} = \frac{N_{\rm up}}{\int_{E_{\rm min}}^{E_{\rm max}}~E_{\nu}^{-2}~{\cal E}_{\rm tot}(E_\nu)~dE_\nu}$
				\\[2mm]
				$N_{\rm up} = 2.39$ (Feldman-Cousins/Conrad)
			\end{alertblock}
		\end{textblock}	
		
		\begin{textblock}{9}(0.5,5.5)
			\begin{block}{Limits}<2->
			\begin{overprint}
				\onslide<2->\centerline{\pgfimage[width=\textwidth]{fig/resultadosAuger/diff_limits_and_models_paper_combined_all_Mod}}
			\end{overprint}
			\end{block}
		\end{textblock}
		
		\begin{textblock}{5.6}(10,8)
			\begin{exampleblock}{Comentarios}<3->
				\begin{itemize}
				\item Cota m\'as estricta presentada por Auger
				\item Inferior a la cota de Waxman-Bahcall
				\item Mayor sensitividad donde se esperan mayores flujos
				\end{itemize}
			\end{exampleblock}
		\end{textblock}
% 		
		\begin{textblock}{5.6}(10,6)
			\begin{alertblock}{}<2->
			\centering
			\tiny
			$\bm{ k_{\rm up}=6.4 \times 10^{-9}~{\rm GeV~cm^{-2}~s^{-1}~sr^{-1}}}$
			\\ a $90\%$ C.L. $1.0\times10^{17}~{\rm eV}<E<2.5\times10^{19.5}~{\rm eV}$
			\end{alertblock}
		\end{textblock}
% % 		\begin{block}{Integrated limit}
% % 		\centering
% % 		$k_{\rm up}=6.54 \times 10^{-9}~{\rm GeV~cm^{-2}~s^{-1}~sr^{-1}}$ at $90\%$ C.L. in $10^{17}~{\rm eV}<E<10^{19.5}~{\rm eV}$
% % 		\end{block}
\end{frame}

\begin{frame}
	\frametitle{Resultados}
		\scriptsize
		\begin{block}{Rate de eventos esperado}
			\begin{center}
				\renewcommand{\arraystretch}{2.0}
				\begin{tabu}{l c c} 
	\hline
	Modelo para       &  Número esperado de eventos     & ~Probabilidad de   \\
	el flujo difuso   &  (1 enero 2004 - 20 junio 2013) & ~observar $0$   \\
	\hline
	\rowfont{\color{red}}
% 	\rowcolor[red]{1.0}
	Cosmogénico - proton, FRII     &  $\sim$ 4.0  & $\sim 1.8\times 10^{-2}$ \\
	\rowfont{\color{red}}
	Cosmogénico - proton, Fermi-LAT &  $\sim$ 3.2  & $\sim 4\times 10^{-2}$   \\
	\rowfont{\color<2->{blue}}
	Cosmogénico - proton, SFR &  $\sim$ 0.9  & $\sim 0.4$               \\
	\rowfont{\color<2->{blue}}
	Cosmogénico - mixto (banda) &  $\sim$ 0.5 $-$ 1.4 & $\sim 0.6~-~0.2$ \\
	\rowfont{\color<3->{Green}}
	Cosmogénico - hierro, FRII &  $\sim$ 0.3  & $\sim 0.7$ \\

	\hline
	\rowfont{\color{red}}
	Astrofísico $\nu$ (AGN) &  $\sim$ 7.2  & $\sim 7\times 10^{-4}$ \\
	\hline
	\end{tabu}
			\end{center}
		\end{block}
		\begin{alertblock}{}
			\begin{itemize}[<+->]
			 \item Flujos optimistas de protones y AGN rechazados $>90\%\rm C.L.$ (\textcolor{Red}{rojos})
			 \item Acerc\'andose a flujos de protones menos optimistas y mixtos (\textcolor{Blue}{azules})
			 \item Todav\'ia lejos de modelos de hierro (\textcolor{Green}{verdes})
			\end{itemize}
		\end{alertblock}
	\end{frame}
% 	
	\begin{frame}{Comentarios finales de la primera parte}
		\begin{center}
		\pgfimage[width=0.75\textwidth]{fig/resultadosAuger/diff_limits_and_many_models_IceCube_data_noextrap}
		\end{center}
		\begin{block}{}
		 \centering
		 \textbf{Desaf\'io para la pr\'oxima generaci\'on de detectores de neutrinos}
		\end{block}
	\end{frame}