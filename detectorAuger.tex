\section[PAO]{Proyecto Auger}

	\begin{frame}
	\frametitle{Observatorio Pierre Auger (PAO)}
	\framesubtitle{Descripci\'on}
		\begin{block}{Detector}
			\begin{center}
				\pgfimage[width=\textwidth]{./fig/detectorAuger/array2}
			\end{center}
		\end{block}
	\end{frame}


	\begin{frame}
	\frametitle{Observatorio Pierre Auger (PAO)}
	\framesubtitle{Detector de superficie}
		\begin{block}{Tanque cherenkov}
			\begin{center}
				\pgfimage[width=0.9\textwidth]{./fig/detectorAuger/sd1}
			\end{center}
		\end{block}
	\end{frame}
	
	\begin{frame}
	\frametitle{Observatorio Pierre Auger (PAO)}
	\framesubtitle{Detector de superficie}
		\begin{block}{Tanque cherenkov}
			\begin{center}
				\pgfimage[width=0.8\textwidth]{./fig/detectorAuger/sd2}
			\end{center}
		\end{block}
	\end{frame}

	\begin{frame}
	\frametitle{?`Como se generan estas se\~nales?}
	\begin{center}
		\begin{block}{}
			\begin{center}
			\pgfimage[width=.65\textwidth]{fig/detectorAuger/Desarrollo_cascada}<1-2>
			\pgfimage[width=.65\textwidth]{fig/detectorAuger/Desarrollo_cascada_2}<3>
			\pgfimage[width=.48\textwidth]{fig/detectorAuger/traza_tot_m}<4>\hspace*{0.01mm}
			\pgfimage[width=.48\textwidth]{fig/detectorAuger/traza_t2_m}<4>
			\end{center}
		\end{block}
		\begin{block}{}<2->
			\begin{center}
			\pgfimage[width=.65\textwidth]{fig/detectorAuger/front_delay_tr}<1>
			\pgfimage[width=.65\textwidth]{fig/detectorAuger/front_delay}<2>
			\pgfimage[width=.65\textwidth]{fig/detectorAuger/front_delay_2}<3->
			\end{center}
		\end{block}
	\end{center}
	\end{frame}
	
	\begin{frame}{Identificaci\'on de neutrinos con el SD de Auger}
		\begin{alertblock}{}\centering
		Con el SD es posible distinguir frentes mu\'onicos de frentes electromagn\'eticos.
% 			With the SD, we can distinguish muonic from electromagnetic shower fronts (using the time structure of the signals in the water Cherenkov stations).
		\end{alertblock}
		
		\begin{block}{Traza Cherenkov medida con una resoluci\'on de 25ns}
		\centering
		\pgfimage[width=0.8\textwidth]{fig/detectorAuger/tanque_muon.pdf}
		\end{block}
		
	\end{frame}

	\begin{frame}{B\'usquedas de neutrinos con le Observatorio}
		\begin{exampleblock}{}\centering
		Existen tres canales de b\'usqueda de neutrinos optimizadas para diferentes rangos angulares.
		\end{exampleblock}
		
		\begin{block}{B\'usquedas:}
		\centering
		\pgfimage[width=\textwidth]{fig/detectorAuger/auger_nu_p}
		\end{block}
		
	\end{frame}
	
	