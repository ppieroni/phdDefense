\documentclass[11pt]{beamer}
\usetheme{Madrid}
\usepackage[utf8]{inputenc}
\usepackage[english]{babel}
\usepackage{amsmath}
\usepackage{amsfonts}
\usepackage{amssymb}
\usepackage{graphicx}
\usepackage{xcolor}
\usepackage[absolute, overlay]{textpos}
\definecolor{AliceBlue}{HTML}{F0F8FF}
\definecolor{AntiqueWhite}{HTML}{FAEBD7}
\definecolor{Aqua}{HTML}{00FFFF}
\definecolor{Aquamarine}{HTML}{7FFFD4}
\definecolor{Azure}{HTML}{F0FFFF}
\definecolor{Beige}{HTML}{F5F5DC}
\definecolor{Bisque}{HTML}{FFE4C4}
\definecolor{Black}{HTML}{000000}
\definecolor{BlanchedAlmond}{HTML}{FFEBCD}
\definecolor{Blue}{HTML}{0000FF}
\definecolor{BlueViolet}{HTML}{8A2BE2}
\definecolor{Brown}{HTML}{A52A2A}
\definecolor{BurlyWood}{HTML}{DEB887}
\definecolor{CadetBlue}{HTML}{5F9EA0}
\definecolor{Chartreuse}{HTML}{7FFF00}
\definecolor{Chocolate}{HTML}{D2691E}
\definecolor{Coral}{HTML}{FF7F50}
\definecolor{CornflowerBlue}{HTML}{6495ED}
\definecolor{Cornsilk}{HTML}{FFF8DC}
\definecolor{Crimson}{HTML}{DC143C}
\definecolor{Cyan}{HTML}{00FFFF}
\definecolor{DarkBlue}{HTML}{00008B}
\definecolor{DarkCyan}{HTML}{008B8B}
\definecolor{DarkGoldenRod}{HTML}{B8860B}
\definecolor{DarkGray}{HTML}{A9A9A9}
\definecolor{DarkGreen}{HTML}{006400}
\definecolor{DarkKhaki}{HTML}{BDB76B}
\definecolor{DarkMagenta}{HTML}{8B008B}
\definecolor{DarkOliveGreen}{HTML}{556B2F}
\definecolor{DarkOrange}{HTML}{FF8C00}
\definecolor{DarkOrchid}{HTML}{9932CC}
\definecolor{DarkRed}{HTML}{8B0000}
\definecolor{DarkSalmon}{HTML}{E9967A}
\definecolor{DarkSeaGreen}{HTML}{8FBC8F}
\definecolor{DarkSlateBlue}{HTML}{483D8B}
\definecolor{DarkSlateGray}{HTML}{2F4F4F}
\definecolor{DarkTurquoise}{HTML}{00CED1}
\definecolor{DarkViolet}{HTML}{9400D3}
\definecolor{DeepPink}{HTML}{FF1493}
\definecolor{DeepSkyBlue}{HTML}{00BFFF}
\definecolor{DimGray}{HTML}{696969}
\definecolor{DodgerBlue}{HTML}{1E90FF}
\definecolor{FireBrick}{HTML}{B22222}
\definecolor{FloralWhite}{HTML}{FFFAF0}
\definecolor{ForestGreen}{HTML}{228B22}
\definecolor{Fuchsia}{HTML}{FF00FF}
\definecolor{Gainsboro}{HTML}{DCDCDC}
\definecolor{GhostWhite}{HTML}{F8F8FF}
\definecolor{Gold}{HTML}{FFD700}
\definecolor{GoldenRod}{HTML}{DAA520}
\definecolor{Gray}{HTML}{808080}
\definecolor{Green}{HTML}{008000}
\definecolor{GreenYellow}{HTML}{ADFF2F}
\definecolor{HoneyDew}{HTML}{F0FFF0}
\definecolor{HotPink}{HTML}{FF69B4}
\definecolor{IndianRed}{HTML}{CD5C5C}
\definecolor{Indigo}{HTML}{4B0082}
\definecolor{Ivory}{HTML}{FFFFF0}
\definecolor{Khaki}{HTML}{F0E68C}
\definecolor{Lavender}{HTML}{E6E6FA}
\definecolor{LavenderBlush}{HTML}{FFF0F5}
\definecolor{LawnGreen}{HTML}{7CFC00}
\definecolor{LemonChiffon}{HTML}{FFFACD}
\definecolor{LightBlue}{HTML}{ADD8E6}
\definecolor{LightCoral}{HTML}{F08080}
\definecolor{LightCyan}{HTML}{E0FFFF}
\definecolor{LightGoldenRodYellow}{HTML}{FAFAD2}
\definecolor{LightGray}{HTML}{D3D3D3}
\definecolor{LightGreen}{HTML}{90EE90}
\definecolor{LightPink}{HTML}{FFB6C1}
\definecolor{LightSalmon}{HTML}{FFA07A}
\definecolor{LightSeaGreen}{HTML}{20B2AA}
\definecolor{LightSkyBlue}{HTML}{87CEFA}
\definecolor{LightSlateGray}{HTML}{778899}
\definecolor{LightSteelBlue}{HTML}{B0C4DE}
\definecolor{LightYellow}{HTML}{FFFFE0}
\definecolor{Lime}{HTML}{00FF00}
\definecolor{LimeGreen}{HTML}{32CD32}
\definecolor{Linen}{HTML}{FAF0E6}
\definecolor{Magenta}{HTML}{FF00FF}
\definecolor{Maroon}{HTML}{800000}
\definecolor{MediumAquaMarine}{HTML}{66CDAA}
\definecolor{MediumBlue}{HTML}{0000CD}
\definecolor{MediumOrchid}{HTML}{BA55D3}
\definecolor{MediumPurple}{HTML}{9370DB}
\definecolor{MediumSeaGreen}{HTML}{3CB371}
\definecolor{MediumSlateBlue}{HTML}{7B68EE}
\definecolor{MediumSpringGreen}{HTML}{00FA9A}
\definecolor{MediumTurquoise}{HTML}{48D1CC}
\definecolor{MediumVioletRed}{HTML}{C71585}
\definecolor{MidnightBlue}{HTML}{191970}
\definecolor{MintCream}{HTML}{F5FFFA}
\definecolor{MistyRose}{HTML}{FFE4E1}
\definecolor{Moccasin}{HTML}{FFE4B5}
\definecolor{NavajoWhite}{HTML}{FFDEAD}
\definecolor{Navy}{HTML}{000080}
\definecolor{OldLace}{HTML}{FDF5E6}
\definecolor{Olive}{HTML}{808000}
\definecolor{OliveDrab}{HTML}{6B8E23}
\definecolor{Orange}{HTML}{FFA500}
\definecolor{OrangeRed}{HTML}{FF4500}
\definecolor{Orchid}{HTML}{DA70D6}
\definecolor{PaleGoldenRod}{HTML}{EEE8AA}
\definecolor{PaleGreen}{HTML}{98FB98}
\definecolor{PaleTurquoise}{HTML}{AFEEEE}
\definecolor{PaleVioletRed}{HTML}{DB7093}
\definecolor{PapayaWhip}{HTML}{FFEFD5}
\definecolor{PeachPuff}{HTML}{FFDAB9}
\definecolor{Peru}{HTML}{CD853F}
\definecolor{Pink}{HTML}{FFC0CB}
\definecolor{Plum}{HTML}{DDA0DD}
\definecolor{PowderBlue}{HTML}{B0E0E6}
\definecolor{Purple}{HTML}{800080}
\definecolor{Red}{HTML}{FF0000}
\definecolor{RosyBrown}{HTML}{BC8F8F}
\definecolor{RoyalBlue}{HTML}{4169E1}
\definecolor{SaddleBrown}{HTML}{8B4513}
\definecolor{Salmon}{HTML}{FA8072}
\definecolor{SandyBrown}{HTML}{F4A460}
\definecolor{SeaGreen}{HTML}{2E8B57}
\definecolor{SeaShell}{HTML}{FFF5EE}
\definecolor{Sienna}{HTML}{A0522D}
\definecolor{Silver}{HTML}{C0C0C0}
\definecolor{SkyBlue}{HTML}{87CEEB}
\definecolor{SlateBlue}{HTML}{6A5ACD}
\definecolor{SlateGray}{HTML}{708090}
\definecolor{Snow}{HTML}{FFFAFA}
\definecolor{SpringGreen}{HTML}{00FF7F}
\definecolor{SteelBlue}{HTML}{4682B4}
\definecolor{Tan}{HTML}{D2B48C}
\definecolor{Teal}{HTML}{008080}
\definecolor{Thistle}{HTML}{D8BFD8}
\definecolor{Tomato}{HTML}{FF6347}
\definecolor{Turquoise}{HTML}{40E0D0}
\definecolor{Violet}{HTML}{EE82EE}
\definecolor{Wheat}{HTML}{F5DEB3}
\definecolor{White}{HTML}{FFFFFF}
\definecolor{WhiteSmoke}{HTML}{F5F5F5}
\definecolor{Yellow}{HTML}{FFFF00}
\definecolor{YellowGreen}{HTML}{9ACD32}


\author{Jaime Alvarez-Mu\~niz\inst{1} \and Pablo Pieroni\inst{2} }
\title{Radio array sensitivity to Earth-Skimming $\nu_\tau$}
\date{February 9 - 13, 2015}


\institute{
	\inst{1} \scriptsize{Departamento de F\'isica de Part\'iculas \& Instituto Galego de F\'isica de Altas Enerx\'ias\\
	Universidad de Santiago de Compostela, Espa\~na.}
	\and
	\inst{2} \scriptsize{Departamento de F\'isica - Facultad de Ciencias Exactas y Naturales\\
	Universidad de Buenos Aires, Argentina.}
	
}

\begin{document}

\begin{frame}
\titlepage
\end{frame}

\begin{frame}{End to end simulation}
		\scriptsize
		\begin{block}{Simulation stages}
			\begin{center}
			\pgfimage[height=0.35\textwidth]{fig/sim_cartoon_upgoing}
			\end{center}
		\end{block}
		\begin{alertblock}{}
			\begin{enumerate}
			 \item ${\rm P(\tau,\nu_\tau)}$: Probability to obtain a $\tau$ with $(E_\tau,\theta)$ given a $\nu_\tau$ with $(E_\nu,\theta)$ using our code
			 \item $\tau$ decay products energy sampling obtained with {\sc TAUOLA}
			 \item Shower evolution and antenna signal calculation with ZHAireS
			 \item Antenna response and efficiency calculation using our code
			\end{enumerate}
		\end{alertblock}
\end{frame}

\begin{frame}{Radio footprint}
		\scriptsize
		\begin{block}{Cherenkov emision projected over the ground}
			\begin{center}
			\pgfimage[width=0.8\textwidth]{fig/coneProy}
			\end{center}
		\end{block}
		\begin{alertblock}{Projection width:}
		\begin{center}
		\begin{displaymath}
			w^2=
			(\tan^2 \theta_{cher}-\tan^2 (\theta-\frac{\pi}{2}))
			(\frac{d}{\sin \theta}-l_{max})^2
			- \tan (\theta-\frac{\pi}{2}) \frac{h_{max}}{\sin \theta} (\frac{d}{\sin \theta}-l_{max})
			- \frac{h_{max}^2}{\sin^2 \theta}
			\end{displaymath}
		\end{center}
		\end{alertblock}
\end{frame}

\begin{frame}{Radio footprint}
		\scriptsize
		\begin{block}{Energy of the tau that goes into shower $\sim1{\rm EeV}$}
			\begin{center}
			\pgfimage[height=0.35\textwidth]{fig/foorPrint_Cone}
			\end{center}
		\end{block}
		\begin{alertblock}{}
			\begin{enumerate}
			 \item Very elongated and narrow footprints
			 \item In agreement with Cherenkov emission prediction
			\end{enumerate}
		\end{alertblock}
\end{frame}


\begin{frame}{Efficiency calculation}
		\scriptsize
		\begin{textblock}{15}(0.5,1.5)
			\begin{block}{\scriptsize Visible energy: $1{\rm EeV}$}
				\begin{center}
				\pgfimage[height=0.3\textwidth]{fig/cores}
				\hfill
				\pgfimage[height=0.3\textwidth]{fig/trigger}
				\end{center}
			\end{block}
		\end{textblock}
		\begin{textblock}{15}(0.5,9.1)
			\begin{alertblock}{}
				\begin{itemize}\scriptsize
				 \item Example: Squared array. 1000 m separation between antennas. 
				 \item Big enough to contain the shower whole.
				 \item 1000 core position in each $(E_{visible},\theta,\tau_{\rm decayHeight})$ bin.
				 \item Antenna response $30-80{\rm Mhz}$/$150-900{\rm Mhz}$.
				 \item Trigger if $n$ antennas above threshold (selected by user)
				 \item Identification efficiency conservatively assumed as 0.9:
					 \begin{itemize}\tiny
					 \item Based on the apparent velocity of the signal and the with of the footprint.
					 \item Not optimized yet
					 \end{itemize}
				\end{itemize}
			\end{alertblock}
		\end{textblock}
\end{frame}

\begin{frame}{Exposure calculation}

		\begin{block}{Formula}
			\begin{center}
			\begin{displaymath}
		\begin{aligned}
			{\cal E} (E_\nu) = 2 \pi 
			\textcolor{Red}{T A}
			\int_{0}^{\infty} 
			\int_{\theta^{cut}}^{\theta^{max}} 
			\int_{0}^{E_\nu} 
			\int_{0}^{E_\tau} 
			\textcolor{Blue}{\epsilon (x_d,\theta,E_{sh})}
			\textcolor{DarkOrange}{
			\frac{e^{-\frac{l(x_d)}{\lambda(E_\tau)}}}{\lambda(E_\tau)}\frac{dl(x_d)}{dx_d}
			}
			\textcolor{DarkViolet}{P(E_{sh}|E_\tau)}\\
			\textcolor{Green}{P(E_\tau|E_\nu,\theta)}
			\textcolor{Magenta}{\sin \theta \cos \theta}
			dE_{sh} dE_\tau  d\theta dx_d
		\end{aligned}
		\end{displaymath}
			\end{center}
		\end{block}
		\begin{exampleblock}{Ingredients}
			\begin{itemize}
			 \item \textcolor{Red}{Time and area}
			 \item \textcolor{Blue}{Trigger and Id efficiency}
			 \item \textcolor{DarkOrange}{Probability of $\tau$ decay at vertical depth $x_d$ in atmosphere.}
			 \item \textcolor{DarkViolet}{Probability of $\tau$ of energy $E_\tau$ producing shower of energy $E_{sh}$}
			 \item \textcolor{Green}{Interaction inside Earth}
			 \item \textcolor{Magenta}{Solid angle}
			\end{itemize}
		\end{exampleblock}

\end{frame}

\begin{frame}{3yr radio array sensitivity}

				\begin{alertblock}{\scriptsize Upper limit calculation}
				\centering
				\scriptsize
				Assume $\nu$ flux: $\Phi_\nu ~ = ~ k ~ E_\nu^{-2}$ $\Rightarrow$ ${\text Sensitivity} = \frac{2.4}{\int E_{\nu}^{-2}~{\cal E}(E_\nu) dE_\nu}$
			\end{alertblock}
		
		
		\begin{block}{\scriptsize Diferential limit: 90000 antennas  in a 900x900m array - Trigger threshold 50${\rm \mu Vm}$/350${\rm \mu Vm}$}
			\begin{center}
			\pgfimage[height=0.45\textwidth]{fig/limits_future}
			\end{center}
		\end{block}
\end{frame}

\end{document}