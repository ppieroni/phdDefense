\documentclass[11pt]{beamer}
\usetheme{Madrid}
\usepackage[utf8]{inputenc}
\usepackage[english]{babel}
\usepackage{amsmath}
\usepackage{amsfonts}
\usepackage{amssymb}
\usepackage{graphicx}
\usepackage{xcolor}
\usepackage[absolute, overlay]{textpos}
\input{htmlColors.tex}

\author{Jaime Alvarez-Mu\~niz\inst{1} \and Pablo Pieroni\inst{2} }
\title{Radio array sensitivity to Earth-Skimming $\nu_\tau$}
\date{February 9 - 13, 2015}


\institute{
	\inst{1} \scriptsize{Departamento de F\'isica de Part\'iculas \& Instituto Galego de F\'isica de Altas Enerx\'ias\\
	Universidad de Santiago de Compostela, Espa\~na.}
	\and
	\inst{2} \scriptsize{Departamento de F\'isica - Facultad de Ciencias Exactas y Naturales\\
	Universidad de Buenos Aires, Argentina.}
	
}

\begin{document}

\begin{frame}
\titlepage
\end{frame}

\begin{frame}{End to end simulation}
		\scriptsize
		\begin{block}{Simulation stages}
			\begin{center}
			\pgfimage[height=0.35\textwidth]{fig/sim_cartoon_upgoing}
			\end{center}
		\end{block}
		\begin{alertblock}{}
			\begin{enumerate}
			 \item ${\rm P(\tau,\nu_\tau)}$: Probability to obtain a $\tau$ with $(E_\tau,\theta)$ given a $\nu_\tau$ with $(E_\nu,\theta)$ using our code
			 \item $\tau$ decay products energy sampling obtained with {\sc TAUOLA}
			 \item Shower evolution and antenna signal calculation with ZHAireS
			 \item Antenna response and efficiency calculation using our code
			\end{enumerate}
		\end{alertblock}
\end{frame}

\begin{frame}{Radio footprint}
		\scriptsize
		\begin{block}{Cherenkov emision projected over the ground}
			\begin{center}
			\pgfimage[width=0.8\textwidth]{fig/coneProy}
			\end{center}
		\end{block}
		\begin{alertblock}{Projection width:}
		\begin{center}
		\begin{displaymath}
			w^2=
			(\tan^2 \theta_{cher}-\tan^2 (\theta-\frac{\pi}{2}))
			(\frac{d}{\sin \theta}-l_{max})^2
			- \tan (\theta-\frac{\pi}{2}) \frac{h_{max}}{\sin \theta} (\frac{d}{\sin \theta}-l_{max})
			- \frac{h_{max}^2}{\sin^2 \theta}
			\end{displaymath}
		\end{center}
		\end{alertblock}
\end{frame}

\begin{frame}{Radio footprint}
		\scriptsize
		\begin{block}{Energy of the tau that goes into shower $\sim1{\rm EeV}$}
			\begin{center}
			\pgfimage[height=0.35\textwidth]{fig/foorPrint_Cone}
			\end{center}
		\end{block}
		\begin{alertblock}{}
			\begin{enumerate}
			 \item Very elongated and narrow footprints
			 \item In agreement with Cherenkov emission prediction
			\end{enumerate}
		\end{alertblock}
\end{frame}


\begin{frame}{Efficiency calculation}
		\scriptsize
		\begin{textblock}{15}(0.5,1.5)
			\begin{block}{\scriptsize Visible energy: $1{\rm EeV}$}
				\begin{center}
				\pgfimage[height=0.3\textwidth]{fig/cores}
				\hfill
				\pgfimage[height=0.3\textwidth]{fig/trigger}
				\end{center}
			\end{block}
		\end{textblock}
		\begin{textblock}{15}(0.5,9.1)
			\begin{alertblock}{}
				\begin{itemize}\scriptsize
				 \item Example: Squared array. 1000 m separation between antennas. 
				 \item Big enough to contain the shower whole.
				 \item 1000 core position in each $(E_{visible},\theta,\tau_{\rm decayHeight})$ bin.
				 \item Antenna response $30-80{\rm Mhz}$/$150-900{\rm Mhz}$.
				 \item Trigger if $n$ antennas above threshold (selected by user)
				 \item Identification efficiency conservatively assumed as 0.9:
					 \begin{itemize}\tiny
					 \item Based on the apparent velocity of the signal and the with of the footprint.
					 \item Not optimized yet
					 \end{itemize}
				\end{itemize}
			\end{alertblock}
		\end{textblock}
\end{frame}

\begin{frame}{Exposure calculation}

		\begin{block}{Formula}
			\begin{center}
			\begin{displaymath}
		\begin{aligned}
			{\cal E} (E_\nu) = 2 \pi 
			\textcolor{Red}{T A}
			\int_{0}^{\infty} 
			\int_{\theta^{cut}}^{\theta^{max}} 
			\int_{0}^{E_\nu} 
			\int_{0}^{E_\tau} 
			\textcolor{Blue}{\epsilon (x_d,\theta,E_{sh})}
			\textcolor{DarkOrange}{
			\frac{e^{-\frac{l(x_d)}{\lambda(E_\tau)}}}{\lambda(E_\tau)}\frac{dl(x_d)}{dx_d}
			}
			\textcolor{DarkViolet}{P(E_{sh}|E_\tau)}\\
			\textcolor{Green}{P(E_\tau|E_\nu,\theta)}
			\textcolor{Magenta}{\sin \theta \cos \theta}
			dE_{sh} dE_\tau  d\theta dx_d
		\end{aligned}
		\end{displaymath}
			\end{center}
		\end{block}
		\begin{exampleblock}{Ingredients}
			\begin{itemize}
			 \item \textcolor{Red}{Time and area}
			 \item \textcolor{Blue}{Trigger and Id efficiency}
			 \item \textcolor{DarkOrange}{Probability of $\tau$ decay at vertical depth $x_d$ in atmosphere.}
			 \item \textcolor{DarkViolet}{Probability of $\tau$ of energy $E_\tau$ producing shower of energy $E_{sh}$}
			 \item \textcolor{Green}{Interaction inside Earth}
			 \item \textcolor{Magenta}{Solid angle}
			\end{itemize}
		\end{exampleblock}

\end{frame}

\begin{frame}{3yr radio array sensitivity}

				\begin{alertblock}{\scriptsize Upper limit calculation}
				\centering
				\scriptsize
				Assume $\nu$ flux: $\Phi_\nu ~ = ~ k ~ E_\nu^{-2}$ $\Rightarrow$ ${\text Sensitivity} = \frac{2.4}{\int E_{\nu}^{-2}~{\cal E}(E_\nu) dE_\nu}$
			\end{alertblock}
		
		
		\begin{block}{\scriptsize Diferential limit: 90000 antennas  in a 900x900m array - Trigger threshold 50${\rm \mu Vm}$/350${\rm \mu Vm}$}
			\begin{center}
			\pgfimage[height=0.45\textwidth]{fig/limits_future}
			\end{center}
		\end{block}
\end{frame}

\end{document}