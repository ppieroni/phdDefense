\section[Exposicion]{C\'alculo de la exposici\'on}

\begin{frame}{C\'alculo de exposici\'on}
\footnotesize
		\begin{block}{F\'ormula}
			\begin{center}
			\begin{displaymath}
			\begin{aligned}
				{\cal E} (E_\nu) = 2 \pi 
				\textcolor{Red}{T A}
				\int_{0}^{\infty} 
				\int_{\theta^{cut}}^{\theta^{max}} 
				\int_{0}^{E_\nu} 
				\int_{0}^{E_\tau} 
				\textcolor<2->{Blue}{\epsilon (x_d,\theta,E_{sh})}
				\textcolor<3->{DarkOrange}{
				\frac{e^{-\frac{l(x_d)}{\lambda(E_\tau)}}}{\lambda(E_\tau)}\frac{dl(x_d)}{dx_d}
				}
				\textcolor<4->{DarkViolet}{P(E_{sh}|E_\tau)}\\
				\textcolor<5->{Green}{P(E_\tau|E_\nu,\theta)}
				\textcolor<6->{Magenta}{\sin \theta \cos \theta}
				dE_{sh} dE_\tau  d\theta dx_d
			\end{aligned}
		\end{displaymath}
			\end{center}
		\end{block}
		\begin{exampleblock}{Ingredientes}
			\begin{itemize}[<+->]
			 \item \textcolor{Red}{Tiempo de medici\'on y \'area del detector.}
			 \item \textcolor{Blue}{Eficiencia de detecci\'on.}
			 \item \textcolor{DarkOrange}{Probabilidad de decaimiento de un $\tau$ a altura ${\rm x_d}$.}
			 \item \textcolor{DarkViolet}{Probabilidad que un $\tau$ de energ\'ia $E_\tau$ produzca una lluvia de energ\'ia $E_{sh}$.}
			 \item \textcolor{Green}{Interacci\'on en la tierra.}
			 \item \textcolor{Magenta}{\'Angulo s\'olido.}
			\end{itemize}
		\end{exampleblock}
\end{frame}

\begin{frame}{Topograf\'ia del detector}
\footnotesize
	\begin{block}{Descripci\'on}
		\begin{center}
		\pgfimage[width=0.85\textwidth]{fig/resultadosRadio/topografia}
		\end{center}
	\end{block}
% 	\begin{block}{}
% 		\begin{center}
% 		
% 		\end{center}
% 	\end{block}
\end{frame}

\begin{frame}{C\'alculo de la eficiencia}
\footnotesize
	\begin{block}{Se lanz\'o cada lluvia 1000 veces sobre la celda primitiva}
	\centering
		\pgfimage[width=0.48\textwidth]{fig/resultadosRadio/17.00_89.90_00.00_00025_01238_1000_1000_90_re}\hspace*{2mm}
		\pgfimage[width=0.48\textwidth]{fig/resultadosRadio/17.00_89.90_00.00_00025_01238_1500_1500_60_re} \\ \vspace*{2mm}
		\pgfimage[width=0.48\textwidth]{fig/resultadosRadio/17.00_89.90_00.00_00025_01238_500_4000_90_de}\hspace*{2mm}
		\pgfimage[width=0.48\textwidth]{fig/resultadosRadio/17.00_89.90_00.00_00025_01238_750_750_60_hc}
	\end{block}
% 	\begin{block}{}
% 		\begin{center}
% 		
% 		\end{center}
% 	\end{block}
\end{frame}

\begin{frame}{C\'alculo de la eficiencia}
\footnotesize
	\begin{block}{Se lanz\'o cada lluvia 1000 veces sobre la celda primitiva}
	\centering
		\pgfimage[width=0.85\textwidth]{fig/resultadosRadio/trigger}
	\end{block}
% 	\begin{block}{}
% 		\begin{center}
% 		
% 		\end{center}
% 	\end{block}
\end{frame}


\begin{frame}{C\'alculo de la eficiencia}
\footnotesize
	\begin{block}{Curvas de eficiencia}
	\centering
		\pgfimage[width=0.85\textwidth]{fig/resultadosRadio/eff50.0_5.0_1000.0_1000.0_60.0_1.0_17.75_m2}
	\end{block}
% 	\begin{block}{}
% 		\begin{center}
% 		
% 		\end{center}
% 	\end{block}
\end{frame}

\begin{frame}{Eficiencias de identificaci\'on}
\footnotesize
	\begin{block}{Eventos DG dejan huellas m\'as anchas que los ES}
	\centering
		\pgfimage[width=0.48\textwidth]{fig/resultadosRadio/foorPrint_ZWv1.34_ntuples_v1.22_Downgoing_phi_90_18.5_89_90_100_5_E}\hspace*{2mm}
		\pgfimage[width=0.48\textwidth]{fig/resultadosRadio/foorPrint_Cone_ZWv1.22_ntuples_v1.21_ChTest_phi_90_18_89.5_90_25_1238_E0_u}
	\end{block}
	
	\begin{block}{Pueden ser discriminados}
	\centering
		\pgfimage[width=0.48\textwidth]{fig/resultadosRadio/idRadio}\hspace*{2mm}
		\pgfimage[width=0.48\textwidth]{fig/resultadosRadio/showerWidth_Comp_DG_ES_Wt}
	\end{block}
	
% 	\begin{block}{}
% 		\begin{center}
% 		
% 		\end{center}
% 	\end{block}
\end{frame}


\section[Resultados]{Resultado final}

\begin{frame}{Comparaci\'on con Auger}
\footnotesize
	\begin{block}{Cociente de cantidad de eventos esperados como funci\'on del threshold local}
	\centering
		\pgfimage[width=0.85\textwidth]{fig/resultadosRadio/CompRadioAuger_1000.0_5.0_1.0_de_modo3}
	\end{block}
	
% 	\begin{block}{}
% 		\begin{center}
% 		
% 		\end{center}
% 	\end{block}
\end{frame}

\begin{frame}{Comparaci\'on con Auger}
\footnotesize
	\begin{block}{Cociente de cantidad de eventos esperados}
	\centering
		\pgfimage[width=0.85\textwidth]{fig/resultadosRadio/CompRadioAuger_50.0_5.0_1.0_de_modo2}
	\end{block}
	
% 	\begin{block}{}
% 		\begin{center}
% 		
% 		\end{center}
% 	\end{block}
\end{frame}

\begin{frame}{Comparaci\'on con Auger}
\footnotesize
	\begin{block}{Tama\~no dle detector}
	\centering
		\pgfimage[width=0.85\textwidth]{fig/resultadosRadio/area/Area_de}
	\end{block}
	
% 	\begin{block}{}
% 		\begin{center}
% 		
% 		\end{center}
% 	\end{block}
\end{frame}

\begin{frame}{Comparaci\'on con Auger}
\footnotesize
	\begin{block}{Tama\~no dle detector}
	\centering
		\pgfimage[width=0.85\textwidth]{fig/resultadosRadio/area/CompRadioAuger_50.0_5.0_1.0_de_modo2}
	\end{block}
	
% 	\begin{block}{}
% 		\begin{center}
% 		
% 		\end{center}
% 	\end{block}
\end{frame}

\begin{frame}{L\'imite diferencial en 3 a\~nos de exposici\'on}

				\begin{alertblock}{\scriptsize C\'alculo del l\'imite}
				\centering
				\scriptsize
				Se asume un flujo: $\Phi_\nu ~ = ~ k ~ E_\nu^{-2}$ $\Rightarrow$ ${\text k (E_\nu)} = \frac{2.4}{\int\limits_{E_\nu-\Delta}^{E_\nu+\Delta} \tilde{E}_{\nu}^{-2}~{\cal E}(\tilde{E}_\nu) d\tilde{E}_\nu}$
			\end{alertblock}
		
		
		\begin{block}{\scriptsize L\'imite diferencial: 90000 antennas - Trigger local 50${\rm \mu Vm}$ - L = 250 ${\rm km}$ - 500${\rm km}$}
			\begin{center}
			\pgfimage[height=0.45\textwidth]{fig/resultadosRadio/limits_future_v1_2_2.pdf}
			\end{center}
		\end{block}
\end{frame}

\begin{frame}{Desempe\~no - L = 250 ${\rm km}$}
	\begin{block}{Rate de eventos}
		\begin{center}
		\renewcommand{\arraystretch}{1.3}
		\footnotesize
		\begin{tabular}{lccc}
			\hline
			\multirow{2}{*}{Modelo} & \multicolumn{3}{c}{Topograf\'ia - \cant{L=250}{km}} \\
			&   Regular &   Panal de abeja &   Bordes densos \\
			\hline
			Cosmogénico - proton, FRII &    45.3 &             52.2 &            52.5 \\
			Cosmogénico - proton, Fermi-LAT &     34.1 &             39.4 &            39.5 \\
			Cosmogénico - proton, SFR &     10.3 &             11.8 &            11.9 \\
			Cosmogénico - H\'ibrido &      5.8 - 14.9 &      6.7 - 17.2 &       6.7 - 17.3 \\
			Cosmogénico - iron, FRII &     3.2 &              3.6 &             3.7 \\
			IceCube extrapolado $E^{-2}$ &      13.1 &             15.1 &            15.2 \\
			IceCube extrapolado \emph{Best fit}  &      12.9 &             14.9 &            15   \\
			\hline
		\end{tabular}
		\end{center}
	\end{block}
\end{frame}

\begin{frame}{Desempe\~no - L = 500 ${\rm km}$}
	\begin{block}{Rate de eventos}
		\begin{center}
		\renewcommand{\arraystretch}{1.3}
		\footnotesize
		\begin{tabular}{lccc}
			\hline
			\multirow{2}{*}{Modelo} & \multicolumn{3}{c}{Topograf\'ia - \cant{L=500}{km}} \\
			&   Regular &   Panal de abeja &   Bordes densos \\
			\hline
			Cosmogénico - proton, FRII &    185.8 &            193   &           191.3 \\
			Cosmogénico - proton, Fermi-LAT &     140.3 &             146.0 &           144.8 \\
			Cosmogénico - proton, SFR &     42.1 &             43.8   &            43.4 \\
			Cosmogénico - H\'ibrido &  23.9 - 61.3 &   24.8 - 63.7 &  24.6 - 63.2 \\
			Cosmogénico - iron, FRII &     13.0   &        13.4 &            13.3 \\
			IceCube extrapolado $E^{-2}$ &      53.6 &         55.5   &            55   \\
			IceCube extrapolado \emph{Best fit} &    52.4 &  53.8  &   53.3 \\
			\hline
		\end{tabular}
		\end{center}
	\end{block}
\end{frame}

