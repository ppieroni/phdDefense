\section[Exposicion]{C\'alculo de la exposici\'on}

\frame{
\begin{block}{}
 \centering
 \Large{
 C\'alculo de la exposici\'on}
\end{block}

}

\begin{frame}{C\'alculo de la exposici\'on}
\footnotesize
		\begin{block}{F\'ormula}
			\begin{center}
			\begin{displaymath}
			\begin{aligned}
				{\cal E} (E_\nu) = 2 \pi 
				\textcolor<4>{Red}{T A}
				\textcolor<4->{DodgerBlue}{
				\int_{0}^{\infty} 
				\int_{\theta^{cut}}^{\theta^{max}} 
				\int_{0}^{E_\nu} 
				\int_{0}^{E_\tau} 
				}
				\textcolor<2->{Green}{P(E_\tau|E_\nu,\theta)}
				\textcolor<2->{DarkOrange}{
				\frac{e^{-\frac{l(x_d)}{\lambda(E_\tau)}}}{\lambda(E_\tau)}\frac{dl(x_d)}{dx_d}
				}
				\textcolor<2->{DarkViolet}{P(E_{sh}|E_\tau)}\\
				\textcolor<3->{Blue}{\epsilon (x_d,\theta,E_{sh})}
				\textcolor<4->{Magenta}{\sin \theta \cos \theta}
				\textcolor<4->{DodgerBlue}{dE_{sh} dE_\tau  d\theta dx_d}
			\end{aligned}
		\end{displaymath}
			\end{center}
		\end{block}
		\begin{exampleblock}{Ingredientes}
			\begin{itemize}
			 \item<2-> \textcolor{Green}{Interacci\'on en la tierra.}
			 \item<2-> \textcolor{DarkOrange}{Probabilidad de decaimiento de un $\tau$ a altura ${\rm x_d}$.}
			 \item<2-> \textcolor{DarkViolet}{Probabilidad que un $\tau$ de energ\'ia $E_\tau$ produzca una lluvia de energ\'ia $E_{sh}$.}
			 \item<3-> \textcolor{Blue}{Eficiencia de detecci\'on.}
			 \item<4-> \textcolor{Magenta}{\'Angulo s\'olido.}
			 \item<4-> \textcolor{DodgerBlue}{Espacio de par\'ametros observado.}
			 \item<4-> \textcolor{Red}{Tiempo de medici\'on y \'area del detector.}
			\end{itemize}
		\end{exampleblock}
		\begin{alertblock}{}<5>
			\centering
			\textbf{Calculo de eficiencias}
		\end{alertblock}
\end{frame}

\begin{frame}{Eficiencias de disparo}
\scriptsize
	
	\begin{exampleblock}{Procedimiento}
	 \begin{enumerate}[<alert@+|+->]
	  \item Definir espacio de par\'ametros 
	  \item Definir topograf\'ia del detector
	  \item Lanzar cada lluvia 1000 veces sobre la celda primitiva
	  \item Verificar disparo (5 antenas sobre el threshold)
	  \item Curvas de eficiencia
	 \end{enumerate}
	\end{exampleblock}
	\begin{overprint}
		\onslide<1>\centerline{\pgfimage[width=0.72\textwidth]{fig/caracterizacionRadio/binesRadio_3}}
		\onslide<2>\centerline{\pgfimage[width=0.72\textwidth]{fig/resultadosRadio/topografia}}
		\onslide<3>\centerline{\pgfimage[width=0.72\textwidth]{fig/resultadosRadio/17.00_89.90_00.00_00025_01238_750_750_60_hc}}
		\onslide<4>\centerline{\pgfimage[width=0.72\textwidth]{fig/resultadosRadio/trigger}}
		\onslide<5>\centerline{\pgfimage[width=0.72\textwidth]{fig/resultadosRadio/eff50.0_5.0_1000.0_1000.0_60.0_1.0_17.75_m2}}
	\end{overprint}
\end{frame}


\begin{frame}{Eficiencias de identificaci\'on}
\footnotesize
	\begin{overprint}
	 \onslide<1>
		\begin{block}{Eventos hadronicos inclinados dejan huellas m\'as anchas que los ES}
		\centering
		\pgfimage[width=0.85\textwidth]{fig/resultadosRadio/idRadio}
		\end{block}
	\onslide<2>
		\begin{block}{Eventos DG dejan huellas m\'as anchas que los ES}
		\centering
		\pgfimage[width=0.85\textwidth]{fig/resultadosRadio/comp_ES_DG/foorPrint_ZWv1.34_ntuples_v1.22_Downgoing_phi_90_18.5_89_90_100_5_E}
		\end{block}
		
	\onslide<3->
		\begin{block}{Eventos hadronicos inclinados dejan huellas m\'as anchas que los ES}
		\centering
		\pgfimage[width=0.85\textwidth]{fig/resultadosRadio/showerWidth_Comp_DG_ES_Wt}
		\end{block}
	\end{overprint}
	\begin{alertblock}{}<4>
	\centering
	\textbf{Se asumi\'o la misma eficiencia que en Auger ($\sim 90\%$)}
	\end{alertblock}
\end{frame}

% \begin{frame}{Eficiencias de identificaci\'on}
% \footnotesize
% 	\begin{block}{Eventos DG dejan huellas m\'as anchas que los ES}
% 	\centering
% 		\pgfimage[width=0.48\textwidth]{fig/resultadosRadio/foorPrint_ZWv1.34_ntuples_v1.22_Downgoing_phi_90_18.5_89_90_100_5_E}\hspace*{2mm}
% 		\pgfimage[width=0.48\textwidth]{fig/resultadosRadio/foorPrint_Cone_ZWv1.22_ntuples_v1.21_ChTest_phi_90_18_89.5_90_25_1238_E0_u}
% 	\end{block}
% 	
% 	\begin{block}{Pueden ser discriminados}
% 	\centering
% 		\pgfimage[width=0.48\textwidth]{fig/resultadosRadio/idRadio}\hspace*{2mm}
% 		\pgfimage[width=0.48\textwidth]{fig/resultadosRadio/showerWidth_Comp_DG_ES_Wt}
% 	\end{block}
% 	
% % 	\begin{block}{}
% % 		\begin{center}
% % 		
% % 		\end{center}
% % 	\end{block}
% \end{frame}


\section[Resultados]{Resultado final}

% \begin{frame}{Desempe\~no respecto de Auger}
% \footnotesize
% 	\begin{block}{Cociente de cantidad de eventos esperados como funci\'on del threshold local}
% 	\centering
% 		\pgfimage[width=0.85\textwidth]{fig/resultadosRadio/CompRadioAuger_1000.0_5.0_1.0_de_modo3}
% 	\end{block}
% % 	\begin{block}{}
% % 		\begin{center}
% % 		
% % 		\end{center}
% % 	\end{block}
% \end{frame}

\begin{frame}{Desempe\~no: Radio vs. Auger}
\footnotesize
	\begin{alertblock}{Comparaci\'on:}\centering
	Cantidad de eventos esperados en el mismo tiempo de medici\'on (3 a\~nos).
	\end{alertblock}
	\begin{block}{Arreglo con bordes densos}<2->
		\begin{overprint}
		\onslide<2-4>
			\centerline{\pgfimage[width=0.85\textwidth]{fig/resultadosRadio/CompRadioAuger_50.0_5.0_1.0_de_modo2_mod}}
		\onslide<5>
			\centerline{\pgfimage[width=0.85\textwidth]{fig/resultadosRadio/CompRadioAuger_50.0_5.0_1.0_de_modo2_mod2}}
		\end{overprint}
	\end{block}

	\begin{exampleblock}{}<3>
	\centering
	\textbf{La exposici\'on aumenta con el \'area}
	\end{exampleblock}
	
	\begin{textblock}{3}(4,6.8)
	\visible<3>{
	\centering\scriptsize
	 \textbf{Par\'ametros:}\\
	 \pgfimage[width=\textwidth]{fig/resultadosRadio/topografia_de}}
	\end{textblock}
	\begin{textblock}{3}(4,6.8)
	\visible<4>{
	\centering\scriptsize
	 \textbf{Area total:}\\
	 \pgfimage[width=\textwidth]{fig/resultadosRadio/topografia_de_area}}
	\end{textblock}
	
\end{frame}


\begin{frame}{L\'imite diferencial en 3 a\~nos de exposici\'on}

				\begin{alertblock}{\scriptsize C\'alculo del l\'imite}
				\centering
				\scriptsize
				Se asume un flujo: $\Phi_\nu ~ = ~ k ~ E_\nu^{-2}$ $\Rightarrow$ ${\text k (E_\nu)} = \frac{2.4}{\int\limits_{E_\nu-\Delta}^{E_\nu+\Delta} \tilde{E}_{\nu}^{-2}~{\cal E}(\tilde{E}_\nu) d\tilde{E}_\nu}$
			\end{alertblock}
		
		
		\begin{block}{\scriptsize L\'imite diferencial: 90000 antennas - Trigger local 50${\rm \mu Vm}$ - L = 250 ${\rm km}$ - 500${\rm km}$}
			\begin{center}
			\pgfimage[height=0.45\textwidth]{fig/resultadosRadio/limits_future_v1_2_2.pdf}
			\end{center}
		\end{block}
\end{frame}

\begin{frame}{Desempe\~no - L = 250 ${\rm km}$ (GRAND)} 
	\begin{block}{Rate de eventos}
		\begin{center}
		\renewcommand{\arraystretch}{1.3}
		\footnotesize
		\begin{tabular}{lccc}
			\hline
			\multirow{2}{*}{Modelo} & \multicolumn{3}{c}{Topograf\'ia - \cant{L=250}{km}} \\
			&   Regular &   Panal de abeja &   Bordes densos \\
			\hline
			Cosmogénico - proton, FRII &    45.3 &             52.2 &            52.5 \\
			Cosmogénico - proton, Fermi-LAT &     34.1 &             39.4 &            39.5 \\
			Cosmogénico - proton, SFR &     10.3 &             11.8 &            11.9 \\
			Cosmogénico - H\'ibrido &      5.8 - 14.9 &      6.7 - 17.2 &       6.7 - 17.3 \\
			\alert{Cosmogénico - iron, FRII} &     \alert{3.2} &              \alert{3.6} &             \alert{3.7} \\
			IceCube extrapolado $E^{-2}$ &      13.1 &             15.1 &            15.2 \\
			IceCube extrapolado \emph{Best fit}  &      12.9 &             14.9 &            15   \\
			\hline
		\end{tabular}
		\end{center}
	\end{block}
	\begin{exampleblock}{}
% 	\begin{itemize}[<alert@+|+->]
	\centering
	 \textbf{Con un arreglo de \cant{\bm{250}}{\bm{km}} es posible estudiar incluso modelos de Fe}
% 	 \item Mezcla con neutrinos IceCube?
% 	\end{itemize}
	\end{exampleblock}
\end{frame}



