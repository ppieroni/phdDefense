\section[Exposicion]{C\'alculo de la exposici\'on}

\begin{frame}{C\'alculo de exposici\'on}
\footnotesize
		\begin{block}{F\'ormula}
			\begin{center}
			\begin{displaymath}
			\begin{aligned}
				{\cal E} (E_\nu) = 2 \pi 
				\textcolor<7>{Red}{T A}
				\textcolor<6->{DodgerBlue}{
				\int_{0}^{\infty} 
				\int_{\theta^{cut}}^{\theta^{max}} 
				\int_{0}^{E_\nu} 
				\int_{0}^{E_\tau} 
				}
				\textcolor<1->{Green}{P(E_\tau|E_\nu,\theta)}
				\textcolor<2->{DarkOrange}{
				\frac{e^{-\frac{l(x_d)}{\lambda(E_\tau)}}}{\lambda(E_\tau)}\frac{dl(x_d)}{dx_d}
				}
				\textcolor<3->{DarkViolet}{P(E_{sh}|E_\tau)}\\
				\textcolor<4->{Blue}{\epsilon (x_d,\theta,E_{sh})}
				\textcolor<5->{Magenta}{\sin \theta \cos \theta}
				\textcolor<6->{DodgerBlue}{dE_{sh} dE_\tau  d\theta dx_d}
			\end{aligned}
		\end{displaymath}
			\end{center}
		\end{block}
		\begin{exampleblock}{Ingredientes}
			\begin{itemize}[<+->]
			 \item \textcolor{Green}{Interacci\'on en la tierra.}
			 \item \textcolor{DarkOrange}{Probabilidad de decaimiento de un $\tau$ a altura ${\rm x_d}$.}
			 \item \textcolor{DarkViolet}{Probabilidad que un $\tau$ de energ\'ia $E_\tau$ produzca una lluvia de energ\'ia $E_{sh}$.}
			 \item \textcolor{Blue}{Eficiencia de detecci\'on.}
			 \item \textcolor{Magenta}{\'Angulo s\'olido.}
			 \item \textcolor{DodgerBlue}{Espacio de par\'ametros observado.}
			 \item \textcolor{Red}{Tiempo de medici\'on y \'area del detector.}
			\end{itemize}
		\end{exampleblock}
		\begin{alertblock}{}<8>
			\centering
			\textbf{Calculo de eficiencias}
		\end{alertblock}
\end{frame}

\begin{frame}{Eficiencias de disparo}
\scriptsize
	
	\begin{exampleblock}{Procedimiento}
	 \begin{enumerate}[<alert@+|+->]
	  \item Definir espacio de par\'ametros 
	  \item Definir topograf\'ia del detector
	  \item Lanzar cada lluvia 1000 veces sobre la celda primitiva
	  \item Verificar disparo
	  \item Curvas de eficiencia
	 \end{enumerate}
	\end{exampleblock}
	\begin{overprint}
		\onslide<1>\centerline{\pgfimage[width=0.72\textwidth]{fig/caracterizacionRadio/binesRadio_3}}
		\onslide<2>\centerline{\pgfimage[width=0.72\textwidth]{fig/resultadosRadio/topografia}}
		\onslide<3>\centerline{\pgfimage[width=0.72\textwidth]{fig/resultadosRadio/17.00_89.90_00.00_00025_01238_750_750_60_hc}}
		\onslide<4>\centerline{\pgfimage[width=0.72\textwidth]{fig/resultadosRadio/trigger}}
		\onslide<5>\centerline{\pgfimage[width=0.72\textwidth]{fig/resultadosRadio/eff50.0_5.0_1000.0_1000.0_60.0_1.0_17.75_m2}}
	\end{overprint}
\end{frame}


\begin{frame}{Eficiencias de identificaci\'on}
\footnotesize
	\begin{overprint}
	 \onslide<1>
		\begin{block}{Eventos DG dejan huellas m\'as anchas que los ES}
		\centering
		\pgfimage[width=0.85\textwidth]{fig/resultadosRadio/idRadio}
		\end{block}
	\onslide<2>
		\begin{block}{Eventos DG dejan huellas m\'as anchas que los ES}
		\centering
		\pgfimage[width=0.85\textwidth]{fig/resultadosRadio/comp_ES_DG/foorPrint_ZWv1.34_ntuples_v1.22_Downgoing_phi_90_18.5_89_90_100_5_E}
		\end{block}
		
	\onslide<3->
		\begin{block}{Eventos DG dejan huellas m\'as anchas que los ES}
		\centering
		\pgfimage[width=0.85\textwidth]{fig/resultadosRadio/showerWidth_Comp_DG_ES_Wt}
		\end{block}
	\end{overprint}
	\begin{alertblock}{}<4>
	\centering
	\textbf{Se asumi\'o la misma eficiencia que en Auger}
	\end{alertblock}
\end{frame}

% \begin{frame}{Eficiencias de identificaci\'on}
% \footnotesize
% 	\begin{block}{Eventos DG dejan huellas m\'as anchas que los ES}
% 	\centering
% 		\pgfimage[width=0.48\textwidth]{fig/resultadosRadio/foorPrint_ZWv1.34_ntuples_v1.22_Downgoing_phi_90_18.5_89_90_100_5_E}\hspace*{2mm}
% 		\pgfimage[width=0.48\textwidth]{fig/resultadosRadio/foorPrint_Cone_ZWv1.22_ntuples_v1.21_ChTest_phi_90_18_89.5_90_25_1238_E0_u}
% 	\end{block}
% 	
% 	\begin{block}{Pueden ser discriminados}
% 	\centering
% 		\pgfimage[width=0.48\textwidth]{fig/resultadosRadio/idRadio}\hspace*{2mm}
% 		\pgfimage[width=0.48\textwidth]{fig/resultadosRadio/showerWidth_Comp_DG_ES_Wt}
% 	\end{block}
% 	
% % 	\begin{block}{}
% % 		\begin{center}
% % 		
% % 		\end{center}
% % 	\end{block}
% \end{frame}


\section[Resultados]{Resultado final}

% \begin{frame}{Desempe\~no respecto de Auger}
% \footnotesize
% 	\begin{block}{Cociente de cantidad de eventos esperados como funci\'on del threshold local}
% 	\centering
% 		\pgfimage[width=0.85\textwidth]{fig/resultadosRadio/CompRadioAuger_1000.0_5.0_1.0_de_modo3}
% 	\end{block}
% % 	\begin{block}{}
% % 		\begin{center}
% % 		
% % 		\end{center}
% % 	\end{block}
% \end{frame}

\begin{frame}{Desempe\~no: Radio vs. Auger}
\footnotesize
	\begin{alertblock}{Comparaci\'on:}\centering
	Cantidad de eventos esperados en mediciones equivalentes.
	\end{alertblock}
	\begin{block}{Arreglo con bordes densos}<2->
	\centering
	\visible<2->{
		\pgfimage[width=0.85\textwidth]{fig/resultadosRadio/CompRadioAuger_50.0_5.0_1.0_de_modo2}}
	\end{block}

	\begin{exampleblock}{}<3>
	\centering
	\textbf{La exposici\'on aumenta con el \'area}
	\end{exampleblock}
	
	\begin{textblock}{3}(4,6.8)
	\visible<3>{
	\centering\scriptsize
	 \textbf{Par\'ametros:}\\
	 \pgfimage[width=\textwidth]{fig/resultadosRadio/topografia_de}}
	\end{textblock}

	
\end{frame}

\begin{frame}{Desempe\~no: Radio vs. Auger}
\footnotesize
	\begin{block}{Tama\~no dle detector}
	\centering
		\pgfimage[width=0.85\textwidth]{fig/resultadosRadio/area/Area_de}
	\end{block}
	
% 	\begin{block}{}
% 		\begin{center}
% 		
% 		\end{center}
% 	\end{block}
\end{frame}

\begin{frame}{Desempe\~no: Radio vs. Auger}
\footnotesize
	\begin{block}{Tama\~no dle detector}
	\centering
		\pgfimage[width=0.85\textwidth]{fig/resultadosRadio/area/CompRadioAuger_50.0_5.0_1.0_de_modo2}
	\end{block}
	
% 	\begin{block}{}
% 		\begin{center}
% 		
% 		\end{center}
% 	\end{block}
\end{frame}

\begin{frame}{L\'imite diferencial en 3 a\~nos de exposici\'on}

				\begin{alertblock}{\scriptsize C\'alculo del l\'imite}
				\centering
				\scriptsize
				Se asume un flujo: $\Phi_\nu ~ = ~ k ~ E_\nu^{-2}$ $\Rightarrow$ ${\text k (E_\nu)} = \frac{2.4}{\int\limits_{E_\nu-\Delta}^{E_\nu+\Delta} \tilde{E}_{\nu}^{-2}~{\cal E}(\tilde{E}_\nu) d\tilde{E}_\nu}$
			\end{alertblock}
		
		
		\begin{block}{\scriptsize L\'imite diferencial: 90000 antennas - Trigger local 50${\rm \mu Vm}$ - L = 250 ${\rm km}$ - 500${\rm km}$}
			\begin{center}
			\pgfimage[height=0.45\textwidth]{fig/resultadosRadio/limits_future_v1_2_2.pdf}
			\end{center}
		\end{block}
\end{frame}

\begin{frame}{Desempe\~no - L = 250 ${\rm km}$}
	\begin{block}{Rate de eventos}
		\begin{center}
		\renewcommand{\arraystretch}{1.3}
		\footnotesize
		\begin{tabular}{lccc}
			\hline
			\multirow{2}{*}{Modelo} & \multicolumn{3}{c}{Topograf\'ia - \cant{L=250}{km}} \\
			&   Regular &   Panal de abeja &   Bordes densos \\
			\hline
			Cosmogénico - proton, FRII &    45.3 &             52.2 &            52.5 \\
			Cosmogénico - proton, Fermi-LAT &     34.1 &             39.4 &            39.5 \\
			Cosmogénico - proton, SFR &     10.3 &             11.8 &            11.9 \\
			Cosmogénico - H\'ibrido &      5.8 - 14.9 &      6.7 - 17.2 &       6.7 - 17.3 \\
			Cosmogénico - iron, FRII &     3.2 &              3.6 &             3.7 \\
			IceCube extrapolado $E^{-2}$ &      13.1 &             15.1 &            15.2 \\
			IceCube extrapolado \emph{Best fit}  &      12.9 &             14.9 &            15   \\
			\hline
		\end{tabular}
		\end{center}
	\end{block}
	\begin{exampleblock}{}
% 	\begin{itemize}[<alert@+|+->]
	\centering
	 \textbf{Con un arreglo de \cant{250}{km} es posible alcanzar modelos de Fe}
% 	 \item Mezcla con neutrinos IceCube?
% 	\end{itemize}

	\end{exampleblock}
\end{frame}

\begin{frame}{Desempe\~no - L = 500 ${\rm km}$}
	\begin{block}{Rate de eventos}
		\begin{center}
		\renewcommand{\arraystretch}{1.3}
		\footnotesize
		\begin{tabular}{lccc}
			\hline
			\multirow{2}{*}{Modelo} & \multicolumn{3}{c}{Topograf\'ia - \cant{L=500}{km}} \\
			&   Regular &   Panal de abeja &   Bordes densos \\
			\hline
			Cosmogénico - proton, FRII &    185.8 &            193   &           191.3 \\
			Cosmogénico - proton, Fermi-LAT &     140.3 &             146.0 &           144.8 \\
			Cosmogénico - proton, SFR &     42.1 &             43.8   &            43.4 \\
			Cosmogénico - H\'ibrido &  23.9 - 61.3 &   24.8 - 63.7 &  24.6 - 63.2 \\
			Cosmogénico - iron, FRII &     13.0   &        13.4 &            13.3 \\
			IceCube extrapolado $E^{-2}$ &      53.6 &         55.5   &            55   \\
			IceCube extrapolado \emph{Best fit} &    52.4 &  53.8  &   53.3 \\
			\hline
		\end{tabular}
		\end{center}
	\end{block}
\end{frame}

