\section{Estrategia}
% Hay otro mecanismo para detectar lluvias atmosfericas (hadronicas o de neutrinos)
% La componente electromagnetica de las lluvias emite ondas de radio
% => se pueden medir con un arreglo de antenas
% En esta parte estudio la factibilidad de buscar neutrinos ES por este metodo 
% Resumen:
% 1. emision de ondas de radio por lluvias atmosfericas 
% 2. simulacion para lluvias hadronicas y para neutrinos ES
% 3. calculo de la eficiencia y la exposicion
% 4. limite esperado y comparacion con Auger y otros experimentos

\begin{frame}
 \frametitle{Motivaci\'on}
 \footnotesize
 \begin{exampleblock}{}
  \centering
  \textbf{Las lluvias atmosf\'ericas emiten se\~nales de radio.}
 \end{exampleblock}

 
 \begin{block}{?`Por qu\'e un arreglo de antenas de radio?}<+->
  \begin{itemize}\setlength\itemsep{2mm}
   \item La amplitud del pulso posee informaci\'on calorim\'etrica $\rightarrow\ E_{prim}$
   \item Los tiempos de arribo guardan informaci\'on sobre la distribuci\'on longitudinal de la lluvia $\rightarrow\ X_{max}$
   \item Casi $100\%$ de tiempo de operaci\'on.
   \item Bajo costo relativo.
  \end{itemize}
 \end{block}
 
 \begin{exampleblock}{GRAND}<+->
  Propone desplegar 90000 antenas en Tianshian, China para desarrollar b\'usquedas de neutrinos c\'osmicos ultra energ\'eticos.
 \end{exampleblock}
 
 \begin{alertblock}{Esta parte de la tesis:}<+->
  Estimar el desempe\~no de un arreglo de 90000 antenas de radio al detectar UHE$\nu$s.
 \end{alertblock}
 
\end{frame}
